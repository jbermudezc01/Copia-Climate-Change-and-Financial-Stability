\documentclass{article}
\usepackage{graphicx} % Required for inserting images
\usepackage{tabularx}
\usepackage{float}
\usepackage{blindtext}
\usepackage{hyperref}


\title{Datos y Metodología Paper}
\author{Juan Pablo Bermudez Cespedes}
\date{Septiembre 2023}

\begin{document}

\maketitle

\section{Data}
Since the purpose of this paper is to study the effects of natural disasters on the mean and variance 
of financial series in emerging economies, we retrieved data from several sources. Firstly, we 
downloaded the 5-year sovereign Credit Default Swap $(CDS)$ daily series for all the countries of 
interest: Brazil, Chile, China, Colombia, Indonesia, South Korea, Malaysia, Mexico, Peru, South Africa, and Turkey. 
The sample we were able to obtain starts on October 10, 2004, and it ends on August 10, 2022. In Table \ref{table:CDS}
 we present the descriptive statistics of the first difference of all the CDS series. \\
\begin{table}[H]
\centering
\small
\begin{tabular}{rrrrrrr}
  \hline
Country & Minimum & Maximum & Mean & Std. Dev. & Skewness & Kurtosis \\ 
  \hline
Brazil & -124.91 & 186.53 & -0.03 & 8.45 & 2.01 & 81.53 \\ 
  Chile & -64.27 & 63.11 & 0.02 & 3.71 & 0.66 & 67.48 \\ 
  China & -58.91 & 67.47 & 0.01 & 3.52 & 0.63 & 73.61 \\ 
  Colombia & -126.52 & 180.59 & -0.02 & 8.20 & 1.38 & 83.91 \\ 
  Indonesia & -223.81 & 324.52 & -0.06 & 13.30 & 3.12 & 170.85 \\ 
  South Korea & -168.55 & 133.28 & 0.00 & 5.72 & -2.92 & 266.79 \\ 
  Malaysia & -101.34 & 119.40 & 0.01 & 5.22 & 1.71 & 128.90 \\ 
  Mexico & -132.96 & 197.20 & 0.01 & 7.31 & 3.82 & 168.49 \\ 
  Peru & -126.10 & 161.39 & -0.04 & 6.72 & 1.82 & 120.92 \\ 
  South Africa & -82.36 & 146.45 & 0.03 & 7.99 & 2.25 & 58.39 \\ 
  Turkey & -131.91 & 166.47 & 0.08 & 10.72 & 1.57 & 48.01 \\ 
  Moving Average & -11.46 & 20.19 & 0.01 & 1.33 & 3.17 & 50.01 \\ 
   \hline
\end{tabular}
\caption{Descriptive Statistics CDS spreads}
\label{table:CDS}
\end{table}
We also aimed to assess the impact of natural disasters on the major stock indices of various countries. 
We utilized the following indices: Bovespa for Brazil, S\&P CLXIPSA for Chile, ChinaA50 for China, 
COLCAP for Colombia, JSX for Indonesia, KOSPI for South Korea, KLCI for Malaysia, S\&P BMVIPC for 
Mexico, IGBVL for Peru, South Africa Top 40 for South Africa, and BIST100 for Turkey. Table 
\ref{table:stock} presents the descriptive statistics of the returns for all these stock index 
series, covering the same sample period as the CDS data, from October 10, 2004, to August 10, 2022. \\

\begin{table}[H]
\centering
\small
\begin{tabular}{rrrrrrr}
  \hline
 Stock Index & Min. & Max. & Mean & Std. Dev. & Skewness & Kurtosis \\ 
  \hline
BIST100 & -11.06 & 12.13 & 0.06 & 1.59 & -0.53 & 7.53 \\ 
  Bovespa & -15.99 & 13.68 & 0.03 & 1.68 & -0.43 & 12.70 \\ 
  ChinaA50 & -9.86 & 9.20 & 0.02 & 1.61 & -0.24 & 7.57 \\ 
  JSX & -10.95 & 7.62 & 0.05 & 1.22 & -0.62 & 10.39 \\ 
  KOSPI & -11.17 & 11.28 & 0.02 & 1.20 & -0.47 & 12.30 \\ 
  S\&P BMVIPC & -7.27 & 10.44 & 0.03 & 1.17 & -0.01 & 9.66 \\ 
  S\&P CLXIPSA & -15.22 & 11.80 & 0.02 & 1.12 & -0.80 & 23.81 \\ 
  SouthAfricaTop40 & -10.45 & 9.11 & 0.04 & 1.29 & -0.20 & 8.76 \\ 
  IGBVL & -117.48 & 230.26 & 0.04 & 4.40 & 23.49 & 1842.97 \\ 
  KLCI & -9.98 & 6.63 & 0.01 & 0.72 & -0.88 & 17.32 \\ 
  COLCAP & -16.29 & 14.69 & 0.03 & 1.25 & -0.80 & 26.72 \\ 
  Moving Average & -1.81 & 0.86 & 0.03 & 0.22 & -1.59 & 12.39 \\ 
   \hline
\end{tabular}
\caption{Descriptive Statistics Stock Indexes Returns}
\label{table:stock}
\end{table}

Since our sample have several countries around the world, we decided to use the international Emergency Events
Database (EM-DAT) published by the Centre for Research on the Epidemiology of Disasters (CRED)\footnote{This database can be 
accesed in the next link \href{https://www.emdat.be/}{https://www.emdat.be/}}

\end{document}
